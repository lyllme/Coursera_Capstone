\section{Introduction}
\subsection{Background}
COVID 19, one of the coronavirus was found in Wuhan China around January and spread to the whole world immediately. It killed hundreds and thousands of people, damages the world economy. How the number of infected people increases is one of the key parameters for everyone who would like to predict future governmental and social policy. For instance, policymakers and managers who had to prepare for the production of medical resources and allocation, restaurants, and shop owners have to calculate how much cash they have, and how long they can keep their business under a curfew of not going out. Predicting the change of infected cases is critical to saving patients, policymakers, and business owners to keep their lives, social stability, and their lives.

\subsection{Potential readers}
Decision-makers such as business leaders, doctors, and policymakers would like to know how patients of coronavirus rise for both medical and business purposes. Others who are seeking their job, students who are watching for their entrance exam, family members who have to take care of other members will also be interested in their decision making.
The target of this research is to expect a theoretical number of infected persons.

\subsection{Nessesity of this research}
At the initial phase in spreading, there is enough supply of test kits to inspect people who want to, 
However, in the later phase, they are distributed to those who have a high possibility of positivity, the government cannot always grasp the actual number of infected people anymore.
I have a strong interest in Japan, where there are not enough test kits as Korea and Singapore. 

As of April 31st, the Japanese government recognizes 14088 cases and 415 Fatalities, which is not as large as Italy and France.
Tn order to estimate the number of potential cases in Japan, I  examine the trend of Italy and the US where the largest number of cases reported as of today and compare the gap between Japan and these two countries.
