\documentclass{article}
\usepackage[utf8]{inputenc}

\title{Covid19}
\author{Shiori Uji}
\date{April 29, 2020}

\usepackage{natbib}
\usepackage{graphicx}
\usepackage{url}

\begin{document}

\maketitle
\section{Introduction}
\subsection{Background}
COVID 19, one of the coronavirus was found in Wuhan China around January and spread to the whole world immediately. It killed hundreds and thousands of people, damages the world economy. How the number of infected people increases is one of the key parameters for everyone who would like to predict future governmental and social policy. For instance, policymakers and managers who had to prepare for the production of medical resources and allocation, restaurants, and shop owners have to calculate how much cash they have, and how long they can keep their business under a curfew of not going out. Predicting the change of infected cases is critical to saving patients, policymakers, and business owners to keep their lives, social stability, and their lives.

\subsection{Business Problem}
Decision-makers such as business leaders, doctors, and policymakers would like to know how patients of coronavirus rise for both medical and business purposes. Others who are seeking their job, students who are watching for their entrance exam, family members who have to take care of other members will also be interested in their decision making. The target of this research is to expect a theoretical number of infected persons.

\subsection{Data set}
The target of this research is to expect a theoretical number of infected persons.Data that might contribute to determining the number of new patients of coronavirus per day include reported cases per day per country and fatalities per day per country. This is a time-series data set, starting from January 21st, 2020. I will use foursquare to acquire locational data to visualize the intercountry difference. All data used for this analysis was taken COVID19 Global Forecasting (Week2) competition held at Kaggle. This train.csv contains the actual number of confirmed cases and fatalities in each countries.

\subsection{Nessesity of this research}
At the initial phase in spreading, there is enough supply of test kits to inspect people who want to, However, in the later phase, they are distributed to those who have a high possibility of positivity, the government cannot always grasp the actual number of infected people anymore. I have a strong interest in Japan, where there are not enough test kits as Korea and Singapore.
As of April 31st, the Japanese government recognizes 14088 cases and 415 Fatalities, which is not as large as Italy and France. Tn order to estimate the number of potential cases in Japan, I examine the trend of Italy and the US where the largest number of cases reported as of today and compare the gap between Japan and these two countries.

\section{Data acquisition and cleaning}
\subsection{Data sources}

(https://www.kaggle.com/c/covid19-global-forecasting-week-2/data).
this train.csv contains the actual number of confirmed cases and fatalities

\subsection{Data cleaning}
There was almost no missing values, however, some countries like US  counts their cases regionally whereas other countries count by country. In order to compare differences in nations, I summed up all the regional data to one country.

\subsection{Feature selection}
After data cleaning, there were hogehoge samples and hogehoge features in the data. 
\section{Exploratory Data analysis}

Calculation of target variable

As of March 31st, over hogehoge countries reported confirmed cases, and hoeghoge countries have fatalities. 

Major countries struggling for Covid-19
graph: heat map of spreading areas

Inter-nation comparison of mortality rate
graph: death rate as of March 31st(cholopleath)

Comparison of 6 countries
graph; comparison of 6 countries
graph: US and ITaly

which represents the main component of the report where you discuss and describe any exploratory data analysis that you did, any inferential statistical testing that you performed, if any, and what machine learnings were used and why.

\section{Predictive modeling}
\subsection{Regression models}
\subsubsection{Applying standard algorithms and their problems}
There are two types of models, regression and classification. In this study, I carried out regression modeling to predict future change of cases.
measure us with sigmoid and exponential
with accumulated and fatalities
?compare the real result and fact?
\subsubsection{Solution to the problems}
\subsubsection{Performances of different models}

\begin{table}[htb]
  \begin{center}
    \caption{title}
    \begin{tabular}{|c|c|c|} \hline
        & Single Linear Regression & Exponential Model\\ \hline
      one & two & three \\
      one & two& three \\
      one & two & three\\ \hline
    \end{tabular}
  \end{center}
\end{table}

\section{Result}

governmental policy

check if model is valid enough?
Italy, with one of Europe’s tightest lockdowns, is preparing for “Phase 2” – an easing affecting much more of daily life than the very limited easing that started on 14 April.
https://www.bbc.com/news/topics/crr7mlg0d2wt/italy

\section{Discussion}
Stay home!

\section{Summary}

\begin{figure}[h!]
\centering
\includegraphics[scale=1.7]{universe}
\caption{The Universe}
\label{fig:universe}
\end{figure}

\section{Conclusion}

and future direction
Comparison of estimates of April and actual data
Use of medical specialized model such as SIR models
Ideas include:
population density, that may affect the speed of spreads
GDP per capita for affordance of medical treatment, that might affect fatalities



``I always thought something was fundamentally wrong with the universe'' \citep{adams1995hitchhiker}ttps://www.kaggle.com/c/covid19-global-forecasting-week-2/data)
\bibliographystyle{plain}
\bibliography{references}
\end{document}
